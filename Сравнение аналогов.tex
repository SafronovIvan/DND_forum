\documentclass[14pt]{extreport}
\setlength{\parindent}{1.25cm}

\usepackage{gost}
\usepackage[unicode, pdftex]{hyperref}
\usepackage{amsmath}
\usepackage{lscape}
\newenvironment{tightcenter}{% центрирование текста
  \setlength\topsep{0pt}
  \setlength\parskip{0pt}
  \begin{center}
}{%
  \end{center}
}
\usepackage{float}
\usepackage{hhline}
\newcommand{\R}{\mathbb{R}}
\newcommand{\N}{\mathbb{N}}
\newcommand{\Z}{\mathbb{Z}}
\setlist[enumerate,1]{label=\arabic*),align=left,leftmargin=*} % Устанавливаем стиль «цифра)« для enumerate
\begin{document}

\begin{table}[H]
\centering
\caption{Сравнение с аналогами}
\label{tab:competitors}
\begin{tabular}{|p{4cm}|p{6cm}|p{6cm}|}
\hline
\multicolumn{1}{|c|}{\textbf{Название}} & \multicolumn{1}{c|}{\textbf{Преимущества}} & \multicolumn{1}{c|}{\textbf{Недостатки}} \\
\hline
\textbf{Fantasy Grounds LFG} &
1) Прямая интеграция с виртуальным столом Fantasy Grounds; 

2) Целевая аудитория опытных игроков и мастеров; 

3) Поддержка множества игровых систем (D\&D, Pathfinder, Savage Worlds). &
1) Требуется покупка лицензии (финансовый барьер); 

2) Устаревший форумный интерфейс, нет удобных фильтров; 

3) Отсутствие инструментов для справедливого отбора игроков. \\
\hline
\textbf{Discord} &
1) Гибкость и бесплатность настройки серверов под сообщества;

 2) Мгновенная коммуникация через голос и чаты; 

3) Огромная и активная аудитория игроков. &
1) Полный хаос в организации записи («гонка за места»); 

2) Отсутствие единого каталога и профилей игроков; 

3) Ручная работа модераторов по управлению списками. \\
\hline
\textbf{RPGTableFinder} &
1) Специализированный сервис для поиска онлайн и оффлайн игр;

 2) Возможность организовать свой стол и управлять расписанием; 

3) Удобный интерфейс с поиском и фильтрами. &
1) Небольшая база пользователей, особенно русскоязычных; 

2) Нет интеграции с популярными виртуальными столами; 

3) Нет алгоритмов для справедливого распределения мест. \\
\hline
\end{tabular}
\end{table}

\section*{Ключевые дифференцирующие факторы разрабатываемой платформы}

Настоящее исследование позволяет идентифицировать системный дефицит на рынке платформ для организации игровых сессий в настольных ролевых играх (НРИ), который выражается в отсутствии механизмов, гарантирующих эгалитарный доступ к игровым ресурсам. Анализ существующих решений (Таблица 1) демонстрирует, что их функциональность ограничивается либо предоставлением коммуникационной инфраструктуры (Discord), либо интеграцией с инструментарием виртуального стола (Fantasy Grounds), либо выполнением роли пассивного агрегатора объявлений (RPGTableFinder). Особенно учитываю то, что некоторые платформы прекратили свою работу на территории Российской Федерации. При этом критически важный аспект — алгоритмическое обеспечение справедливого распределения игровых мест — остаётся нереализованным.

Предлагаемая платформа вводит парадигмальный сдвиг, позиционируя себя не как ещё один инструмент поиска, а как \textbf{система управления справедливым доступом}. Её архитектурным ядром является детерминированный алгоритм расчёта приоритетов, основанный на временнóм интервале, прошедшем с момента последнего участия пользователя в сеансе. Данный подход устраняет ключевые недостатки аналогов:

\begin{itemize}
    \item \textbf{Ликвидация хаотичности.} В отличие от моделей, основанных на принципе «первый откликнувшийся» (preemptive response model), что характерно для текстовых чатов и простых досок объявлений, предложенный алгоритм замещает стохастический процесс детерминированной очередью с приоритетом. Это исключает ситуационное неравенство, возникающее из-за разницы в скорости реакции пользователей.
    \item \textbf{Автоматизация и снижение транзакционных издержек.} Платформа автоматизирует процессы формирования состава игроков, уведомления и ведения очереди ожидания, тем самым минимизируя организационную нагрузку на мастера, которая в текущих решениях ложится на него в виде ручного администрирования.
    \item \textbf{Создание устойчивой экосистемы.} Алгоритм приоритетов выполняет не только утилитарную, но и социальную функцию, выступая в роли механизма ротации участников. Он предотвращает формирование закрытых групп и систематически предоставляет доступ новым или менее активным участникам, что способствует долгосрочному росту и оздоровлению сообщества.
    \item \textbf{Обеспечение прозрачности.} Процесс принятия решения о распределении мест становится полностью формализованным и прозрачным для всех сторон. Каждый пользователь обладает информацией о критериях отбора и своём положении в очереди, что повышает уровень доверия к платформе и снижает потенциальные конфликты.
\end{itemize}

Таким образом, дифференцирующим фактором разрабатываемого решения является переход от пассивного посредничества к активному управлению распределением игровых сессий на основе формальных правил справедливости. Это позволяет устранить фундаментальное противоречие между ограниченностью игровых ресурсов (мест на партию у мастера) и неограниченным спросом на них со стороны сообщества, предлагая технологически опосредованную модель социального контракта внутри игровой экосистемы.

\end{document}